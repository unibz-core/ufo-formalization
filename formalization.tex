\documentclass{article}

\usepackage{listings}
\usepackage{xcolor}

\definecolor{codegreen}{rgb}{0,0.6,0}
\definecolor{codegray}{rgb}{0.5,0.5,0.5}
\definecolor{codepurple}{rgb}{0.58,0,0.82}
\definecolor{backcolour}{rgb}{0.95,0.95,0.92}

\lstdefinestyle{mystyle}{
    backgroundcolor=\color{backcolour},
    commentstyle=\color{codegreen},
    keywordstyle=\color{magenta},
    numberstyle=\tiny\color{codegray},
    stringstyle=\color{codepurple},
    basicstyle=\ttfamily\footnotesize,
    breakatwhitespace=false,
    breaklines=true,
    keepspaces=true,
    numbers=left,
    numbersep=5pt,
    showspaces=false,
    showstringspaces=false,
    showtabs=false,
    tabsize=2
}

\lstset{style=mystyle}

\title{A TPTP Formalization of the Unified Foundational Ontology}
\author{
    Daniele Porelo,
    Jo\~ao Paulo A. Almeida,
    Giancarlo Guizzardi,\\
    Claudenir M. Fonseca,
    Tiago Prince Sales
}
\date{\today}

\begin{document}
\maketitle

\begin{abstract}
This document presents a formalization of the Unified Foundation Ontology (UFO) expressed in first-order logics through the TPTP syntax. This formalization is intended to support verification of UFO's theory through automated provers and consistency checkers.
\end{abstract}

\section{Introduction}

This document presents a formalization of the Unified Foundation Ontology (UFO) expressed in first-order logics through the TPTP syntax. This formalization is intended to support verification of UFO's theory through automated provers and consistency checkers.

\section{UFO's TPTP Specification}

\lstinputlisting{ufo_2021.tex}

% \bibliographystyle{abbrv}
% \bibliography{main}

\end{document}
% This is never printed